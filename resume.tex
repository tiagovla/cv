\documentclass[11pt]{cv}

\name{Tiago Vilela Lima Amorim}
\address{Belo Horizonte, MG, Brazil}
\email{tiagovla@ufmg.br}
\phone{+5531991599886}

\geometry{
    top=20mm,
    left=20mm,
    right=20mm,
    bottom=20mm
}

\begin{document}


\makeprofileheader

\begin{Objective}
    Electrical Engineering PhD with expertise in numerical analysis
    and scientific computing, focused on developing computational
    methods for solving complex engineering problems. Seeking to
    contribute advanced modeling and simulation skills in research or
    industry environments.
\end{Objective}

\begin{Education}[https://lattes.cnpq.br/9062605056335996]
    \educationitem
    {Federal University of Minas Gerais}
    {Belo Horizonte, Brazil}
    {Bachelor of Electrical Engineering (Power Electronics and Power Systems)}
    {2011 -- 2016}
    {Undergraduate Thesis}
    {Frequency Response Modeling of Electric Circuits Using Digital Systems}
    \educationitem
    {University of Illinois at Urbana-Champaign}
    {Urbana-Champaign, USA}
    {Bachelor of Electrical Engineering (Science without Borders,
    CAPES Scholarship)}
    {2014 -- 2014}
    {}
    {}
    \educationitem
    {Federal University of Minas Gerais}
    {Belo Horizonte, Brazil}
    {Master of Electrical Engineering (CAPES Scholarship)}
    {2018 -- 2020}
    {Dissertation}
    {MLPG-MoM Hybrid Numerical Technique Applied to Electromagnetic Scattering}
    \educationitem
    {Federal University of Minas Gerais}
    {Belo Horizonte, Brazil}
    {PhD Program in Electrical Engineering (CAPES Scholarship)}
    {Dec. 2020 -- Jun. 2025}
    {Thesis}
    {Discontinuous Galerkin Method Applied to Electromagnetic Problems}
    \educationitem
    {The Ohio State University}
    {Columbus, USA}
    {Sandwich PhD Program (PDSE, CAPES Scholarship)}
    {Sept. 2023 -- May. 2024}
    {}
    {}
\end{Education}

\begin{Experience}
    \experienceitem
    {Scientific Research Intern — Wireless Ad Hoc Networks}
    {2012}
    {Federal University of Minas Gerais}
    {Belo Horizonte, Brazil}
    {
    \item Investigated routing protocols and network topology
        strategies for decentralized wireless communication.
    }
    \experienceitem
    {Teaching Assistant — Electric Circuits}
    {2013}
    {Federal University of Minas Gerais}
    {Belo Horizonte, Brazil}
    {
    \item Assisted in teaching undergraduate electric circuit theory,
        including DC and AC analysis.
    \item Helped students with problem-solving during lab sessions
        and provided support with course material.
    }
    \experienceitem
    {Visiting Schollar — Computational Electromagnetics}
    {Sept. 2023 -- May. 2024}
    {The Ohio State University, ElectroScience Laboratory}
    {Columbus, OH, USA}
    {
    \item Developed high-performance time-domain solvers for
        electromagnetic wave propagation.
    \item Integrated numerical models for dispersive media into an
        in-house simulation framework.
    }
\end{Experience}

\begin{Skills}
    \skillcategory{Technical Proficiency}{
        \skillitem{\textbf{Proficient:} Python, C, C++, Lua, CMake,
        Git, \LaTeX, Linux, Matlab, Docker, CI/CD.}
        \skillitem{\textbf{Familiar:} Shell scripting,
            Ansible, Assembly, Reverse engineering, SQL, NoSQL, Rust, Go,
            JavaScript, TypeScript, Astro, React, Android development,
        Cloud computing, Cloud infrastructure.}
    }
    \skillcategory{Electrical Engineering Software}{
        \skillitem{PSpice, Multisim, Cadence OrCAD, COMSOL
        Multiphysics, AutoCAD, Inkscape}
    }
    \skillcategory{Linguistic Proficiency}{
        \skillitem{Portuguese (Native), English (C1, TOEFL ITP:
        602), French (A2), Spanish (A1).}
    }
\end{Skills}

\begin{Projects}[https://github.com/tiagovla]
    \projectitem{Plane Wave Expansion Framework for Photonic Crystal
    Analysis}{https://github.com/tiagovla/morpho.py}
    \projectitem{FDTD Framework for Electromagnetic
    Simulations}{https://github.com/tiagovla/fdtd.py}
    \projectitem{Analytical Solutions for Electromagnetic
    Scattering}{https://github.com/tiagovla/scatsol}
    \projectitem{DGTD Framework for Electromagnetic
    Simulations}{https://github.com/tiagovla/oiseau}
    \projectitem{LaTeX Package for Generating Professional Curricula
    Vitae}{https://github.com/tiagovla/cv}
    \projectitem{Contributions to FEniCSx (Open-source PDE solving
    platform)}{https://github.com/FEniCS}
\end{Projects}

\begin{Publications}
    \publicationitem
    {\textbf{AMORIM, T. V. L.}; MOREIRA, F. J. S.; RESENDE, U. C..
        Aplicação de Técnica Sem-Malha Híbrida na Solução de Espalhamento
    Eletromagnético}
    {In: XXXVII Brazilian Symposium on Telecommunications and Signal
    Processing (SBRT), 2019, Petrópolis}
    \publicationitem
    {\textbf{AMORIM, T. V. L.}; SILVA, E. J.; MOREIRA, F. J. S..
        Discontinuous Galerkin Time-Domain Method in Computational
    Electrodynamic Scattering Problems}
    {In: XX Brazilian Symposium on Microwave and Optoelectronics
    (SBMO), 2022, Natal}
    \publicationitem
    {\textbf{AMORIM, T. V. L.}; SILVA, E. J.; MOREIRA, F. J. S.;
        TEIXEIRA, F. L.. A Plug-and-Play DGTD Implementation for General
    Dispersive Media}
    {2024 IEEE AP-S International Symposium on Antennas and
    Propagation, Firenze, Italy, July 2024}
    \publicationitem
    {\textbf{AMORIM, T. V. L.}; SILVA, E. J.; MOREIRA, F. J. S.;
        TEIXEIRA, F. L.. Modular Discontinuous Galerkin Time-Domain for
    General Dispersive Media with Vector Fitting}
    {IEEE Journal on Multiscale and Multiphysics Computational
    Techniques, vol. 10, pp. 179--186, Feb. 2025}
\end{Publications}

\end{document}
