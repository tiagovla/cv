\documentclass{cv}

\name{Tiago Vilela Lima Amorim}
\address{Belo Horizonte, MG, Brazil}
\email{tiagovla@ufmg.br}
\phone{+55 31991599886}

\begin{document}

\makeprofileheader

\begin{Objective}
    Research in numerical analysis and scientific computing.
\end{Objective}

\begin{Education}[https://lattes.cnpq.br/9062605056335996]
    \educationitem
    {Federal University of Minas Gerais}
    {Belo Horizonte, Brazil}
    {Bachelor of Electrical Engineering (Power Electronics and Power Systems)}
    {2011 -- 2016}
    {Undergraduate Thesis}
    {Frequency Response Modeling of Electric Circuits Using Digital Systems}
    \educationitem
    {University of Illinois at Urbana-Champaign}
    {Urbana-Champaign, USA}
    {Bachelor of Electrical Engineering (Science without Borders,
    CAPES Scholarship)}
    {2014 -- 2014}
    {}
    {}
    \educationitem
    {Federal University of Minas Gerais}
    {Belo Horizonte, Brazil}
    {Master of Electrical Engineering (CAPES Scholarship)}
    {2018 -- 2020}
    {Dissertation}
    {MLPG-MoM Hybrid Numerical Technique Applied to Electromagnetic Scattering}
    \educationitem
    {Federal University of Minas Gerais}
    {Belo Horizonte, Brazil}
    {PhD Program in Electrical Engineering (CAPES Scholarship)}
    {Dec. 2020 -- Jun. 2025}
    {Thesis}
    {Discontinuous Galerkin Method Applied to Electromagnetic Problems}
    \educationitem
    {The Ohio State University}
    {Columbus, USA}
    {Sandwich PhD Program in Electrical Engineering (PDSE, CAPES Scholarship)}
    {Sept. 2023 -- May. 2024}
    {}
    {}
\end{Education}

\begin{Experience}
    \academicitem{Scientific Research Scholarship in Wireless Ad Hoc
    Network}{2012}{2012}{Federal University of Minas Gerais}
    \academicitem{Teaching Assistant Scholarship in Electric
    Circuits}{2013}{2013}{Federal University of Minas Gerais}
\end{Experience}

\begin{Publications}
    \publicationitem
    {\textbf{AMORIM, T. V. L.}; MOREIRA, F. J. S.; RESENDE, U. C..
        Aplicação de Técnica Sem-Malha Híbrida na Solução de Espalhamento
    Eletromagnético}
    {In: XXXVII Brazilian Symposium on Telecommunications and Signal
    Processing (SBRT), 2019, Petrópolis}
    \publicationitem
    {\textbf{AMORIM, T. V. L.}; SILVA, E. J.; MOREIRA, F. J. S..
        Discontinuous Galerkin Time-Domain Method in Computational
    Electrodynamic Scattering Problems}
    {In: XX Brazilian Symposium on Microwave and Optoelectronics
    (SBMO), 2022, Natal}
    \publicationitem
    {\textbf{AMORIM, T. V. L.}; SILVA, E. J.; MOREIRA, F. J. S.;
        TEIXEIRA, F. L.. A Plug-and-Play DGTD Implementation for General
    Dispersive Media}
    {2024 IEEE AP-S International Symposium on Antennas and
    Propagation, Firenze, Italy, July 2024}
    \publicationitem
    {\textbf{AMORIM, T. V. L.}; SILVA, E. J.; MOREIRA, F. J. S.;
        TEIXEIRA, F. L.. Modular Discontinuous Galerkin Time-Domain for
    General Dispersive Media with Vector Fitting}
    {IEEE Journal on Multiscale and Multiphysics Computational
    Techniques, vol. 10, pp. 179--186, Feb. 2025}
\end{Publications}

\section{Skills}
Technical Proficiency \vspace{-6pt}
\begin{itemize}
        \resitem{\textbf{Proficient:} Python, Lua, C, C++, CMake,
        Matlab, Git, \LaTeX.}
        \resitem{\textbf{Familiar:} Linux (Usage and Shell
            Scripting), Docker, Ansible, Reverse Engineering, Javascript,
        SQL, NoSQL, Rust. }
\end{itemize}
Linguistic Proficiency \vspace{-6pt}
\begin{itemize}
        \resitem{Portuguese (Native), English (TOEFL ITP\@: 602),
        Spanish (Basic), and French (Basic).}
\end{itemize}

\projectssection{https://github.com/tiagovla}
\begin{itemize}[leftmargin=10pt, itemsep=0pt]
    \item[--] Plane Wave Expansion Framework for Photonic Crystal Analysis
        $|$
        \emph{\href{https://github.com/tiagovla/morpho.py}{\color{blue}GitHub}}
    \item[--] FDTD Framework for Electromagnetic Simulations
        $|$
        \emph{\href{https://github.com/tiagovla/fdtd.py}{\color{blue}GitHub}}
    \item[--] Analytical Solutions for Electromagnetic Scattering
        $|$
        \emph{\href{https://github.com/tiagovla/scatsol}{\color{blue}GitHub}}
    \item[--] Contributions to FEniCSx (Open Source Computing
        Platform for Solving PDEs)
        $|$ \emph{\href{https://github.com/FEniCS}{\color{blue}GitHub}}
\end{itemize}

\end{document}
