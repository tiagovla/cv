\documentclass{cv}

\begin{document}

\name{Tiago Vilela Lima Amorim}
\vspace{0.2cm}
\contact{Av. Pres. Antônio Carlos, 6627}{31270--901 Belo
    Horizonte}{Brazil}{tiagovla@ufmg.br}{+55031991599886}


\section{Objective}
\hspace{1pt}\parbox{0.99\textwidth}{
    Research in numerical analysis and scientific computing.
}
%
\vspace{-5pt}
%
\educationsection{https://lattes.cnpq.br/9062605056335996}

\educationsubsection{Federal University of Minas Gerais}{Belo Horizonte, Brazil}
{Bachelor of Electrical Engineering (Emphasis in Power Electronics and Power Systems)}
{2011--2016}
{Undergraduate Thesis}
{Frequency Response Modeling of Electric Circuits Using Digital Systems}
\\
\edsubsectionshort{University of Illinois at Urbana-Champaign}
{Urbana-Champaign, USA}
{Bachelor of Electrical Engineering (Science without Borders, CAPES Scholarship)}
{2014--2014}
\\
\educationsubsection{Federal University of Minas Gerais}
{Belo Horizonte, Brazil}{Master of Electrical Engineering (CAPES Scholarship)}
{2018--2020}
{Dissertation}
{MLPG-MoM Hybrid Numerical Technique Applied to Electromagnetic
    Scattering}\vspace{0.1cm}
\\
\educationsubsection{Federal University of Minas Gerais}{Belo Horizonte,
    Brazil}{PhD Program in Electrical Engineering (CAPES
    Scholarship)}
{2021--current}
{Thesis}
{Discontinuous Galerkin Method Applied to
    Electromagnetic Problems}\vspace{0.1cm}
\vspace{-10pt}

\section{Academic Experience}
\begin{itemize}[leftmargin=10pt, noitemsep]
    \item[\textbf{--}] Scientific Research Scholarship in Wireless Ad Hoc Network (2012--2012), Federal University of Minas Gerais.
    \item[\textbf{--}] Teaching Assistent Scholarship in Electric Circuits (2013--2013), Federal University of Minas Gerais.
\end{itemize}

\section{Publications}
\begin{itemize}[leftmargin=10pt]
    \item[\textbf{--}] \textbf{AMORIM, T. V. L.}; SILVA, E. J.; MOREIRA, F. J. S.. Discontinuous Galerkin Time-Domain Method in Computational Electrodynamic Scattering Problems. In: XX Brazilian Symposium on Microwave and Optoelectronics (SBMO), 2022, Natal.
    \item[\textbf{--}] \textbf{AMORIM, T. V. L.}; MOREIRA, F. J. S.; RESENDE, U. C.. Aplicação de Técnica Sem-Malha Híbrida na Solução de Espalhamento Eletromagnético. In: XXXVII Brazilian Symposium on Telecommunications and Signal Processing (SBRT), 2019, Petrópolis.
\end{itemize}

\section{Skills}
Technical Proficiency \vspace{-6pt}
\begin{itemize}
    \resitem{\textbf{Proficient:} Python, Lua, C, C++, CMake, Matlab, Git, \LaTeX.}
    \resitem{\textbf{Familiar:} Linux (Usage and Shell Scripting), Docker, Ansible, Reverse Engineering, Javascript, SQL, NoSQL, Rust. }
\end{itemize}
Linguistic Proficiency \vspace{-6pt}
\begin{itemize}
    \resitem{Portuguese (Native), English (TOEFL ITP\@: 602), Spanish (Basic), and French (Basic).}
\end{itemize}

\projectssection{https://github.com/tiagovla}
\begin{itemize}[leftmargin=10pt, itemsep=0pt]
    \item[--] Plane Wave Expansion Framework for Photonic Crystal Analysis
        $|$ \emph{\href{https://github.com/tiagovla/morpho.py}{\color{blue}GitHub}}
    \item[--] FDTD Framework for Electromagnetic Simulations
        $|$ \emph{\href{https://github.com/tiagovla/fdtd.py}{\color{blue}GitHub}}
    \item[--] Analytical Solutions for Electromagnetic Scattering
        $|$ \emph{\href{https://github.com/tiagovla/scatsol}{\color{blue}GitHub}}
    \item[--] Contributions to FEniCSx (Open Source Computing Platform for Solving PDEs)
        $|$ \emph{\href{https://github.com/FEniCS}{\color{blue}GitHub}}
\end{itemize}

\end{document}
